\thispagestyle{plain}
\begin{center}
	\large
	%\textbf{ABSTRACT}
	\vspace{0.5cm}	
\end{center}

        With the advancement in field of technology, various techniques have been developed with the aim of improving the interactivity in the education system. As classrooms may contain a large number of students each from diverse environment, an effective interaction tool is a necessity these days. The same situation may also arise in the conferences. Video projection is in widespread use for multimedia presentations in classrooms and in conferences. 

A particular application is the interactive demonstration of software with a computer whose screen content is sent to a video beamer. An uncomfortable aspect here is that the usual keyboard/mouse computer limits the possibilities of the speakers by tying them to the location of the computer with its devices of interaction. To avoid this restriction, we have developed a system using a common laser pointer tracked by a video camera as an input device. Video cameras already present in multimedia lecture rooms can be used for this purpose, which reduces the required overhead compared to special tracking devices, like electro-magnetic ones. Compared to video-based gesture recognition or tracking of a pointing stick, video-tracking of a laser point is less sensitive to variations in the ambient
light.

The project is, technically, divided into two parts – the software part and the hardware part. The hardware part deals with the generation of the pulses at the specified interval to cause the laser to blink with a particular frequency whereas the software part deals with the detection of the laser pointer on the projected screen and the movement of the mouse according to the movement of the laser pointer.

The laser point on the screen is captured by a video camera, and its location recognized by image processing techniques. The behavior of the point is translated into signals sent to the mouse input of the computer causing the same reactions as if they came from the mouse. More complex interaction paradigms are composed from the elementary operations ”switch on/off” and pointing of the laser pen.

